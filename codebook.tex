\documentclass{article}\usepackage[]{graphicx}\usepackage[dvipsnames]{xcolor}
% maxwidth is the original width if it is less than linewidth
% otherwise use linewidth (to make sure the graphics do not exceed the margin)
\makeatletter
\def\maxwidth{ %
  \ifdim\Gin@nat@width>\linewidth
    \linewidth
  \else
    \Gin@nat@width
  \fi
}
\makeatother

\definecolor{fgcolor}{rgb}{0.345, 0.345, 0.345}
\newcommand{\hlnum}[1]{\textcolor[rgb]{0.686,0.059,0.569}{#1}}%
\newcommand{\hlsng}[1]{\textcolor[rgb]{0.192,0.494,0.8}{#1}}%
\newcommand{\hlcom}[1]{\textcolor[rgb]{0.678,0.584,0.686}{\textit{#1}}}%
\newcommand{\hlopt}[1]{\textcolor[rgb]{0,0,0}{#1}}%
\newcommand{\hldef}[1]{\textcolor[rgb]{0.345,0.345,0.345}{#1}}%
\newcommand{\hlkwa}[1]{\textcolor[rgb]{0.161,0.373,0.58}{\textbf{#1}}}%
\newcommand{\hlkwb}[1]{\textcolor[rgb]{0.69,0.353,0.396}{#1}}%
\newcommand{\hlkwc}[1]{\textcolor[rgb]{0.333,0.667,0.333}{#1}}%
\newcommand{\hlkwd}[1]{\textcolor[rgb]{0.737,0.353,0.396}{\textbf{#1}}}%
\let\hlipl\hlkwb

\usepackage{framed}
\makeatletter
\newenvironment{kframe}{%
 \def\at@end@of@kframe{}%
 \ifinner\ifhmode%
  \def\at@end@of@kframe{\end{minipage}}%
  \begin{minipage}{\columnwidth}%
 \fi\fi%
 \def\FrameCommand##1{\hskip\@totalleftmargin \hskip-\fboxsep
 \colorbox{shadecolor}{##1}\hskip-\fboxsep
     % There is no \\@totalrightmargin, so:
     \hskip-\linewidth \hskip-\@totalleftmargin \hskip\columnwidth}%
 \MakeFramed {\advance\hsize-\width
   \@totalleftmargin\z@ \linewidth\hsize
   \@setminipage}}%
 {\par\unskip\endMakeFramed%
 \at@end@of@kframe}
\makeatother

\definecolor{shadecolor}{rgb}{.97, .97, .97}
\definecolor{messagecolor}{rgb}{0, 0, 0}
\definecolor{warningcolor}{rgb}{1, 0, 1}
\definecolor{errorcolor}{rgb}{1, 0, 0}
\newenvironment{knitrout}{}{} % an empty environment to be redefined in TeX

\usepackage{alltt}
\title{Pacifism and Antimilitarism in Japan}
\author{Jozef Rivest}
\date{ }

\usepackage[margin=3cm]{geometry}

\usepackage[authordate, backend=biber]{biblatex-chicago}
\addbibresource{references.bib}

\usepackage{graphicx}
\graphicspath{{../images}}

\usepackage[dvipsnames]{xcolor}
\usepackage{hyperref}
\hypersetup{
  colorlinks=true,
  linkcolor=Maroon,
  urlcolor=Maroon,
  citecolor=Maron
}

\usepackage{listings}

\usepackage{tabularray}
\usepackage{float}
\usepackage{graphicx}
\usepackage{rotating}
\usepackage[normalem]{ulem}
\UseTblrLibrary{booktabs}
\UseTblrLibrary{siunitx}
\newcommand{\tinytableTabularrayUnderline}[1]{\underline{#1}}
\newcommand{\tinytableTabularrayStrikeout}[1]{\sout{#1}}
\NewTableCommand{\tinytableDefineColor}[3]{\definecolor{#1}{#2}{#3}}
\IfFileExists{upquote.sty}{\usepackage{upquote}}{}
\begin{document}
\maketitle
\tableofcontents



\clearpage

\section{Introduction}

The present document introduce the codebook 
for the following dataset. For an exemple of 
the entry in the dataset, please refer to the 
code chunk below. 

\begin{knitrout}\footnotesize
\definecolor{shadecolor}{rgb}{0.969, 0.969, 0.969}\color{fgcolor}\begin{kframe}
\begin{alltt}
\hlkwd{print}\hldef{(mil_bases)}
\end{alltt}
\begin{verbatim}
##     region      city operated personel personel_prop type size prop_size longitude latitude
## 1 Kanagawa  Yokosuka        2    24000          0.06    0  568      0.02  35.29249 139.6696
## 2   Aomori Hachinohe        0       NA            NA    2   NA        NA  40.55782 141.4656
\end{verbatim}
\end{kframe}
\end{knitrout}


\section{Region}

\verb|region| here refers to the administrative region 
where the military base is located. The term region 
is employed here as the denomination depends for each 
country. See the list below to understand what \verb|region|
corresponds to for each country present in the dataset. 

\begin{itemize}
  \item Japan $\rightarrow$ Prefectures
\end{itemize}

\section{City}

\verb|city| corresponds to the city where the base is situated. 

\section{Operated}

\verb|operated| refers to the country who is operating the base.
It can takes three values: Japan, United States, or Joint.

\begin{table}[H]
\centering
\begin{tblr}[         %% tabularray outer open
]                     %% tabularray outer close
{                     %% tabularray inner open
colspec={Q[]Q[]},
row{1}={}{font=\bfseries,},
cell{2-3}{2}={}{halign=c,},
}                     %% tabularray inner close
\toprule
Type & Value \\ \midrule %% TinyTableHeader
Naval Base & 0 \\
Ground Base & 1 \\
Air Base & 2 \\
\bottomrule
\end{tblr}
\end{table} 

\section{Personnel}

The \verb|personnel| is the total number of personnel recensed 
to be working at the base.

\section{Personnel prop.}

The \verb|personnel_prop| is the ratio of the working personnel 
at the base with the total population of the city. 

\section{Type}

The \verb|type| refers to the type of forces the base is 
hosting. As bases can host more than one group forces, 
a numerical value is assigned to each to account also 
for combinations. 

\begin{table}[H]
\centering
\begin{tblr}[         %% tabularray outer open
]                     %% tabularray outer close
{                     %% tabularray inner open
colspec={Q[]Q[]},
row{1}={}{font=\bfseries,},
cell{2-3}{2}={}{halign=c,},
}                     %% tabularray inner close
\toprule
Type & Value \\ \midrule %% TinyTableHeader
Naval Base & 0 \\
Ground Base & 1 \\
\bottomrule
\end{tblr}
\end{table} 

\section{Size}

\verb|personel| corresponds to the total number of working personel 
officially recensed. 

\section{Proportion Size}

The \verb|prop_size| variable corresponds to the ratio of the 
total area of the base over the total area of the city where the 
base is located. The estimation is obtained by dividing to size of 
the base in hectars by the total size of the city in hectars. 

\section{Longitude}

The \verb|longitude| variable corresponds to the longitude location
of the base. Positive value indicates the location north of the
equator line, and negative values means location south of the equator
line.

\section{Latitude}

The \verb|latitude| is a variable that locates the latitude 
location of the base. 

\end{document}
