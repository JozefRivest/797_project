\documentclass[12pt]{article}
\title{Pacifism and Antimilitarism in Japan}
\author{Jozef Rivest}
\date{ }

\usepackage[margin=2.5cm]{geometry}

\usepackage[authordate, backend=biber]{biblatex-chicago}
\addbibresource{references.bib}

\usepackage{graphicx}
\graphicspath{{../images}}

\usepackage[dvipsnames]{xcolor}
\usepackage{hyperref}
\hypersetup{
  colorlinks=true,
  linkcolor=Maroon,
  urlcolor=Maroon,
  citecolor=Maroon
}

\begin{document}
\maketitle

\begin{abstract}
  Insert the abstract here
\end{abstract}

\section{Introduction}

\begin{itemize}
  \item Why is this dataset useful for political scientists?
  \item Why is this dataset useful for my own research? 
  \item Why seems feasible for this semester?
\end{itemize}

This dataset seeks to introduce information about U.S. 
military bases in East asian countries. A growing body 
of literature has been interested about the role and the 
effect U.S. military bases and locations play on the host 
countries, especially how it shapes citizen's attitudes. 
For example, \cite{jakobsen_jakobsen19} seeked to understand 
how U.S. military deployments in allied countries lead to 
a growing sentiment of ``free--riding'' among the host 
country over a certain threshold. Furthermore, \cite{allen_etal20} 
found that increased contact with U.S. military troops 
and the economic benefits generating from hosting these 
bases has a positive effect on attitudes. 

However, these findings are also challenged by empirical 
evidences that go against these previous claims. 
\textcite{hikotani_etal23} challenge the argument from Allen 
and al. by highlighting that their argument greatly overlook 
the disproportionate concentration of U.S. troops in some areas.
Using the case of Okinawa in Japan, they find that the people 
in this prefecture are more likely to have contact with U.S. 
personnel and troops, and yet they hold more \textit{negative}
views toward them. \textcite{horiuchi_tago23} also found clear 
evidences of this ``Not-In-My-Backyard'' attitudes in Japan 
against the U.S. military. Japanese do value the alliance, 
but do not want the deployment of these arms in their 
vicinity. These findings of negative views toward U.S. military 
base go beyond the case of Japan, as movements global movments
known as ``No Bases'' have surge in South Korea, Italy, and 
Puerto Rico \parencite{vine19}.

Yet, no dataset has been created yet to help us understading popular
views of the U.S. military, and how does it differs with the
interaction with local military. Furthermore, among the previous
design some are also survey experiement (e.g., \textcite{horiuchi_tago23}),
while other are historical account of the evolution of the protest
movements (e.g., \textcite{vine19}) or how these movements failed
to achieve their goal (e.g., \parencite{kim17a}). While this is
not a negative thing per se, we still lack a comprehensive dataset
widely available that would allow more finegrained and contemporary
analyses of this phenomenom.


\section{Literature Review}


\section{Data and Methods}

The objective of this paper is to introduce a dataset that contains
information about U.S. military bases abroad and local military
bases. The goal for the semester will be to create a dataset
containing informations about US military bases and JSDF bases.

Each row of the dataset will be one base. As an illustration, the 

\clearpage
\printbibliography
\clearpage

\appendix
\section{Appendix}

Insert preliminary codebook. 

\end{document}
