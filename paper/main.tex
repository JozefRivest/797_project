\documentclass[12pt]{article}
\title{Pacifism and Antimilitarism in Japan: Introducing a novel
dataset}
\author{Jozef Rivest}
\date{ }

\usepackage[margin=2.5cm]{geometry}

\usepackage[authordate, backend=biber]{biblatex-chicago}
\addbibresource{references.bib}

\usepackage{graphicx}
\graphicspath{{../images}}

\usepackage[dvipsnames]{xcolor}
\usepackage{hyperref}
\hypersetup{
  colorlinks=true,
  linkcolor=Maroon,
  urlcolor=Maroon,
  citecolor=Maroon,
  filecolor=Maroon
}

\usepackage{tabularray}
\usepackage{float}
\usepackage{graphicx}
\usepackage{rotating}
\usepackage[normalem]{ulem}
\UseTblrLibrary{booktabs}
\UseTblrLibrary{siunitx}
\newcommand{\tinytableTabularrayUnderline}[1]{\underline{#1}}
\newcommand{\tinytableTabularrayStrikeout}[1]{\sout{#1}}
\NewTableCommand{\tinytableDefineColor}[3]{\definecolor{#1}{#2}{#3}}

\begin{document}
\maketitle

\begin{abstract}
  Insert the abstract here
\end{abstract}

\section{Introduction}

% \begin{itemize}
%   \item Why is this dataset useful for political scientists?
%   \item Why is this dataset useful for my own research? 
%   \item What seems feasible for this semester?
% \end{itemize}

This dataset would also be highly valuable for my own research as
I am interested in political attitudes toward the military and the
state in a comparative fashion. More precisely, I am interested
about how historical legacies -- such as wars, colonialism, and
state and nation building strategies -- interact with contemporary
challenges in shapping citizen's attitudes toward certain issue,
using East and Southeast Asia as case studies. I define issue here
broadly, as states are currently undergoing a lot of challenges.

For this project, I want to limit myself to the attitudes toward
the military. There have been a rise in international issues for
the last decade. In East Asia, more precisely, the nuclearization
of North Korea and the growing assertiveness of China in both the
South China Sea and East China Sea have lead to a weakening in the
regional balance of power. My goal is to understand how people react
to these changes. I also want to capture the heterogeneity of these 
attitudes and the factors that lead to different views and perspectives 
given the regional security environment. For this, and as it will 
be presented in the coming paragraphs, understanding the interaction 
between the population with both the local and the U.S. military 
will be theoretically and empirically important. 

This dataset seeks to introduce information about U.S. military
bases in East asian countries. A growing body of literature has
been interested about the role and the effect U.S. military bases
and locations play on the host countries, especially how it shapes
citizen's attitudes.  For example, \cite{jakobsen_jakobsen19} seeked
to understand how U.S. military deployments in allied countries
lead to a growing sentiment of ``free--riding'' among the host
country over a certain threshold. Furthermore, \cite{allen_etal20}
found that increased contact with U.S. military troops and the
economic benefits generating from hosting these bases has a positive
effect on attitudes.

However, these findings are also challenged by empirical evidences
that go against these previous claims.  \textcite{hikotani_etal23}
challenge the argument from Allen and al. by highlighting that their
argument greatly overlook the disproportionate concentration of
U.S. troops in some areas.  Using the case of Okinawa in Japan,
they find that the people in this prefecture are more likely to
have contact with U.S.  personnel and troops, and yet they hold
more \textit{negative} views toward them. \textcite{horiuchi_tago23}
also found clear evidences of this ``Not-In-My-Backyard'' attitudes
in Japan against the U.S. military. Japanese do value the alliance,
but do not want the deployment of these arms in their vicinity.
These findings of negative views toward U.S. military base go beyond
the case of Japan, as movements global movments known as ``No Bases''
have surge in South Korea, Italy, and Puerto Rico \parencite{vine19}.

Yet, no dataset has been created yet to help us understading popular
views of the U.S. military, and how does it differs with the
interaction with local military. Furthermore, among the previous
design some are also survey experiement (e.g., \textcite{horiuchi_tago23}),
while other are historical account of the evolution of the protest
movements (e.g., \textcite{vine19}) or how these movements failed
to achieve their goal (e.g., \parencite{kim17a}). While this is not
a negative thing per se, we still lack a comprehensive dataset
widely available that would allow more finegrained and contemporary
analyses of this phenomenom.

For this semester, I will collect data about both the U.S. and Japan's 
military facilities located in Japan. I want the dataset to provide 
measures as fine grained as possible, so it becomes easier and more 
robust to study empirically the interaction between the local 
citizens and the different military forces. 

\section{Literature Review}


\section{Data and Methods}

The objective of this paper is to introduce a dataset that contains
information about U.S. military bases abroad and local military
bases. The goal for the semester will be to create a dataset
containing informations about US military bases and JSDF bases.
The main goal of the dataset is to introduce a multidimensionnal 
overview of the different military facilities in Japan, for both 
the U.S. and Japan. I say multidimensionnal because there's a great 
heterogeneity across all facilities. 

This can be disagregated on two levels. The first one concerns the
\textit{absolute} measurement of the bases themselves. For example,
they vary in terms of size, the number of working personel, and the
type of forces it hosts. Hence, several indicators can be used to
disagregate the bases, and compare them on many individual scales.
The second one concerns the \textit{relative} measurement of the
bases. As the size of the base varies, so does the size of the vicinity 
where they are located. Hence, they do not occup the space in the same 
way. Furthermore, the prefecture and the city where they are located 
also vary in size. accounting for these variation is important as 
the previous studies do not necessarily agree on whether the contact 
with the military personel increase or not positive attitudes. 

Finally, previous studies focused exclusively on U.S. military, 
and almost not on Japan's facilities. While \textcite{horiuchi_tago23}
did include Japan operated facilities in their experimental design 
it does not capture the real preferences of citizens living near 
these bases. Hence, we lack this knowledge for now, as previous 
studies failed to account for the description of this relationship 
in the context of living close to Japan's military facilities. 

Each row of the dataset will be one base. As an illustration, please 
see the tentative codebook in the \href{appendix}{appendix}. Several 
indicators will be employed to measure both the absolute and relative 
size of each facility, while also categorizing them based on the type 
of forces they host and who operate it. 

To validate this novel dataset, and especially on how it can captures 
heterogeneity in attitudes toward the military and defense issues, 
we will use data from an original survey fielded in April 2025 in 
Japan with Rakuten Insight. A national representative sample based on 
age, gender, and location was collected (n=1,733). In this survey, 
four questions were asked about Japan's security. The questions and 
their respective scale can be find in the Table \ref{tbl:datago-questions}.

\begin{table}[h]
\begin{talltblr}[
  caption={Questions about Japan's security},
  label={tbl:datago-questions}
  ]{
    colspec={X[]X[]},
    row{1}={}{cmd=\textbf}
}
\toprule
Questions & Scale \\ \midrule
The government should spend more in the military and defense. & Strongly disagree, Disagree, Agree, Strongly agree\\
Japan should have the capacity to preemptively attack ennemy missile bases. & Strongly disagree, Disagree, Agree, Strongly agree\\
Some have been discussing to amend the Article 9 of the constitution to newly stipulate the existence of the Self-Defense Forces, while keeping paragraphs 1 and 2 unchanged. Do you agree with these amendments to Article 9? Are you against it? & Strongly disagree, Disagree, Agree, Strongly agree\\
Japan's military power should be further strengthened & Strongly disagree, Disagree, Agree, Strongly agree \\
\bottomrule
\end{talltblr}
\end{table}

Furthemore, as many of these questions were taken from the Todai-Asahi 
survey it would be possible to merge some of these dataset together and 
track the evolution trough time, which could help dissantangle many 
interacting factors and alternative explanations. 

\clearpage
\printbibliography
\clearpage

\appendix
\section{Appendix}
\label{appendix}

Insert preliminary codebook. 

\end{document}
